\documentclass[11pt]{report}
\usepackage[italian]{babel}
\begin{document}
\pagestyle{empty}
\parindent 0pt
\hsize = 14.0truecm

\newcommand{\dx}{{\partial \over \partial x}}
\newcommand{\dxx}{{\partial^2 \over \partial x^2}}
\newcommand{\dt}{{\partial \over \partial t}}


{\bf  \begin{center} {\large \bf Eq. Calore 1D }  \end{center} }

$$
   \dt T  = K \dxx T   \label{1} 
$$


Dove $T(t,x)$ rappresenta la temperatura e  $K$ rappresenta la conduzione.

Discretizziamo lo spazio ed il tempo: $ t_n= n\Delta t$ e $ x_j=j \Delta x$.

{\bf  \begin{center} {\large \bf Schema 1 }  \end{center} }

Utilizziamo il seguente schema centrato nello spazio (secondo ordine in $\Delta x$) ed Eulero nel tempo (primo ordine in $\Delta t$).

$$
T^{n+1}_j = T^n_j + \sigma ( T^n_{j-1} - 2 T^n_j +  T^n_{j+1})    \label{2} 
$$

Dove $  \sigma= K{ \Delta t / {\Delta x}^2}$. Per studiare la stabilit\'a di questo schema possiamo studiare la crescita di una perturbazione. Assumiamo di conoscere la soluzione esatta $T_E$ e sostituiamo $ T = T_E + \epsilon $. sviluppando l'errore $\epsilon$ con fourier poniamo:

$$ \epsilon (t,x) = A(t) e^{ikx} $$ 

sostituendo in Eq. \label(2) otteniamo:

$$
A^{n+1} = ( 1+2 \sigma (cos(k\Delta x) - 1) A^n
$$

$A$ non cresce nel tempo se $ | 1-2 \sigma (1-cos(k\Delta x)  | < 1 $ ossia se $ \sigma < 1/2 $


{\bf  \begin{center} {\large \bf Schema 2 }  \end{center} }

Utilizziamo il seguente schema centrato nello spazio e nel tempo ( secondo ordine in $\Delta x$ e in $\Delta t$).

$$
T^{n+1}_j = T^{n-1}_j + 2\sigma ( T^n_{j-1} - 2 T^n_j +  T^n_{j+1})    \label{3} 
$$


per studiare la stabilit\'a di questo schema possiamo studiare la crescita di una perturbazione. Assumiamo di conoscere la soluzione esatta $T_E$ e sostituiamo $ T = T_E + \epsilon $. sviluppando l'errore $\epsilon$ con fourier poniamo:

$$ \epsilon (t,x) = A(t) e^{ikx} $$ 

sostituendo in Eq. \label(2) otteniamo:

$$
A^{n+1} =    A^{n-1}  +4 \sigma (cos(k\Delta x) - 1) A^n
$$

La soluzione di questa equazione alle differenze pu\'o essere trovata assumendo $A^n = p^n$ ossia p elevato alla n.
Raccogliendo $p^{n-1}$ otteniamo:

$$
p^2 - 4 \sigma (cos(k\Delta x) - 1) p  -1 =0;
$$

Osserviamo che il prodotto delle radici vale -1 e non \'e possibile avere entrambe le soluzioni in modulo minori di 1.

{\bf  \begin{center} {\large \bf Schema 3 }  \end{center} }

Utilizziamo il seguente schema centrato nello spazio (secondo ordine in $\Delta x$) ed Eulero nel tempo (primo ordine in $\Delta t$).

$$
T^{n+1}_j = T^n_j + \sigma ( T^{n+1}_{j-1} - 2 T^{n+1}_j +  T^{n+1}_{j+1})    \label{4} 
$$

Osserviamo che la derivata seconda \'e calcolata al tempo $n+1$. Studiamo la stabilit\'a di questo schema introducendo un piccolo errore:
$$ \epsilon (t,x) = A(t) e^{ikx} $$ 

sostituendo in Eq. \label(4) otteniamo:

$$
A^{n+1} = A^n / ( 1+2 \sigma (1-cos(k\Delta x) ) 
$$

$A$ non cresce nel tempo se $ | 1+2 \sigma (1-cos(k\Delta x)  | > 1 $ ossia per ogni  $ \sigma  $.


Per risolverla numericamente riscriviamo la \label{4} come:

$$
    T^{n+1}_{j-1} - (2+1/\sigma) T^{n+1}_j   +  T^{n+1}_{j+1}  = -(1 /  \sigma)T^n_j   \label{5} 
$$

Occorre prestare attenzione a come si applicano le condizioni al contorno.

\end{document}
